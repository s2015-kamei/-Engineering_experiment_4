\documentclass[a4j,dvipdfmx,uplatex,11pt]{jsarticle}
\usepackage{expreport}
\usepackage{graphicx}
\usepackage[truedimen]{geometry}

\geometry{verbose,a4paper,tmargin=2cm,bmargin=2cm,lmargin=2cm,
 rmargin=2cm,headheight=0.5cm,headsep=0.5cm,footskip=0.8cm}

\usepackage[deluxe]{otf}
\usepackage[noto-otc,unicode]{pxchfon}

\subject{工 学 実 験 Ⅳ}
\thema{コンピュータグラフィックス}
\teachers{プロハースカ、高石、永田}
\period{令和4年 10月6日 〜 11月17日}
\deadline{令和4年 11月28日 (月曜日)}
\pnumber{01} %出席番号
\author{情報 太郎} %氏名


\begin{document}
 \maketitle
 

\tableofcontents 
 
\newpage 

\section{実験の目的}

...


\section{課題2}


リスト記述の例をリスト\ref{src:example}に示す。

\begin{lstlisting}[style=pov,mathescape=true,caption={POV-Rayコードの例}, label={src:example}]
#for( i, 0, 89)
    object{
        sphere { <1, 0, -2 > , 0.1}
        pigment{color Orange }
        finish{phong 1}
        rotate <0, 0, 12*i>
        translate <0, 0, 0.05*i>
    }
#end
\end{lstlisting}


 
\end{document}
